\documentclass[11pt,a4paper]{report}
\usepackage[ngerman]{babel} % deutsch und deutsche Rechtschreibung
\usepackage[utf8]{inputenc} % Unicode Text 
\usepackage[T1]{fontenc} % Umlaute und deutsches Trennen
\usepackage{textcomp} % Sonderzeichen wie Euro, Copyright, etc.
\usepackage[hyphens]{url} % Hyperlinks, eMail-Adressen, Pfadangaben
\usepackage{amssymb} % mathmatische Symbole
\usepackage{microtype} % reguliert Abstände zwischen Buchenstaben
\usepackage{graphicx} % wir wollen Bilder einfügen
\usepackage{float} % Umgebung die sich automatisch im Dokument an passenden Positionen bewegen (floaten)
\usepackage{emptypage} % Wirklich leer bei leeren Seiten
\usepackage{paralist} % Spezielle Aufzählungstypen wie z.B. begin{compactenum}[a)] https://de.wikibooks.org/wiki/LaTeX-W%C3%B6rterbuch:_Aufz%C3%A4hlung
\usepackage[german=quotes]{csquotes} % Für deutsche Anführungsstriche
\usepackage{hyperref} % macht ToC und alle internen Lonks klickbar (für leichtere Navigation)
\hypersetup{
    colorlinks,
    citecolor=black,
    filecolor=black,
    linkcolor=black,
    urlcolor=black
}

\usepackage{listings} % schöne Quellcode-Listings
% ein paar Einstellungen für akzeptable Listings
\lstset{basicstyle=\ttfamily, columns=[l]flexible, mathescape=true, showstringspaces=false, numbers=left, numberstyle=\tiny}
\lstset{language=C++} % Set your language (you can change the language for each code-block optionally)
% more information about listings: https://en.wikibooks.org/wiki/LaTeX/Source_Code_Listings

% Spezialpakete
\usepackage{epigraph} % Zum Positionieren von Bemerkungen links, rechts, oben und unten vom Text
\setlength{\epigraphrule}{0pt} % kein Trennstrich

% Seitenlayout
\usepackage[paper=a4paper,width=14cm,left=35mm,height=22cm]{geometry}
\usepackage{setspace}
\linespread{1.15}
\setlength{\parskip}{0.5em}
\setlength{\parindent}{0em} % im Deutschen Einrückung nicht üblich, leider

\newcommand{\phv}{\fontfamily{phv}\fontseries{m}\fontsize{9}{11}\selectfont}
\usepackage{fancyhdr} % ermöglicht schickere Header und Footer
\pagestyle{fancy}
\renewcommand{\chaptermark}[1]{\markboth{#1}{}}
\fancyhead[L]{\phv \leftmark}
\fancyhead[RE,LO]{\phv \nouppercase{\leftmark}}
\fancyhead[LE,RO]{\phv \thepage}
% Unten besser auf alles Verzichten
%\fancyfoot[L]{\textsf{\small \kurztitel}}
\fancyfoot[C]{\ } % keine Seitenzahl unten
%\fancyfoot[R]{\textsf{\small Medieninformatik}}

% Quellen aufteilen z.B. in Online-Quellen und Literaturverzeichnis
\usepackage{bibtopic} 

% Config 1: Times New Roman, gewohnter Font, ok tt und serifenlos
%\usepackage{mathptmx} 
%\usepackage[scaled=.95]{helvet}
%\usepackage{courier}

% Config 2: Palatino mit guten Fonts für tt und serifenlos
\usepackage{mathpazo} % Palatino, mal was anderes
\usepackage[scaled=.95]{helvet}
\usepackage{palatino}

% Config 3: New Century Schoolbook sieht auch nett aus (macht auch tt und serifenlos)
%\usepackage{newcent}

% Mehr Informationen zu Fonts: https://de.sharelatex.com/learn/Font_typefaces


% Zum Zeigen von Fehlern
\usepackage{soulutf8}
\newcommand*\falsch{\st}

% damit wir nicht so viel tippen müssen, nur für Demo 
%\usepackage{blindtext} 

% Float-Objekte sollen die (Sub)Section nicht verlassen in der sie eingefügt worden sind
\usepackage[section]{placeins}

% Für den Befehl \FloatBarrier - Damit kann man Grenzen für Float-Objekte erstellen, die sie nicht verlassen dürfen.
\usepackage{placeins}

% Für farbige Schriften 
\usepackage{color}

\definecolor{mygreen}{rgb}{0,0.6,0}
\definecolor{mygray}{rgb}{0.5,0.5,0.5}
\definecolor{mymauve}{rgb}{0.58,0,0.82}

\lstset{ %
  backgroundcolor=\color{white},   % choose the background color; you must add \usepackage{color} or \usepackage{xcolor}
  basicstyle=\footnotesize\ttfamily,        % the size of the fonts that are used for the code
  breakatwhitespace=false,         % sets if automatic breaks should only happen at whitespace
  breaklines=true,                 % sets automatic line breaking
  captionpos=b,                    % sets the caption-position to bottom
  commentstyle=\color{mygreen},    % comment style
  deletekeywords={...},            % if you want to delete keywords from the given language
	otherkeywords={...},           % if you want to add more keywords to the set
  escapeinside={\%*}{*)},          % if you want to add LaTeX within your code
  extendedchars=true,              % lets you use non-ASCII characters; for 8-bits encodings only, does not work with UTF-8
  frame=none,	                   % adds a frame around the code
  keepspaces=true,                 % keeps spaces in text, useful for keeping indentation of code (possibly needs columns=flexible)
  keywordstyle=\color{blue},       % keyword style
  language=C++,                    % the language of the code
  numbers=left,                    % where to put the line-numbers; possible values are (none, left, right)
  numbersep=10pt,                   % how far the line-numbers are from the code
  numberstyle=\footnotesize, % the style that is used for the line-numbers
  rulecolor=\color{black},         % if not set, the frame-color may be changed on line-breaks within not-black text (e.g. comments (green here))
  showtabs=false,                  % show tabs within strings adding particular underscores
  stepnumber=1,                    % the step between two line-numbers. If it's 1, each line will be numbered
  tabsize=1,	                   % sets default tabsize to 2 spaces
}

%\lstdefinestyle{foo}{
%  moredelim=[is][\color{red}\underbar]{@}{@}
%}

\begin{document}


\begin{titlepage}
  \begin{center}
    % Kopf der Seite
    \parbox[t]{14cm}{
		\centering
      {\fontfamily{ppl}\selectfont
			\Large
				Technische Hochschule Nürnberg Georg Simon Ohm \\
				Fakultät Informatik
			}
	}
	\\[3cm]
    \vfill    
    {\fontfamily{ppl}\selectfont \huge IT-Projektbericht} \\[0.5cm]
    %{\large im Rahmen des Moduls IT-Projekt} \\[5mm]
    %\rule{\textwidth}{1pt}\\[0.5cm]
    {\fontfamily{ppl}\selectfont \LARGE \bfseries OHMComm \\ Entwicklung eines plattformunabhängiges Framework zur Audioübertragung}
    %\rule{\textwidth}{1pt}    
    \vfill
    {
			\fontfamily{ppl}\selectfont
      Verfasser \\ \large Daniel Stadelmann \\ Jonas Ziegler \\ Kamal Bhatti \\ \mbox{ } \\
      \normalsize Datum \\ \large \today \\ \mbox{ } \\
      \normalsize Betreuer \\ \large Prof. Dr. M. Teßmann \\
    } 
\end{center}
\end{titlepage}

\cleardoublepage

% Zusammenfassung
\begin{abstract} 
OHMComm ist ein plattformunabhängiges Audiokommunikationsframework, dass im Rahmen des IT-Projekts an der Technischen Hochschule Nürnberg entwickelt wurde. Das Ziel des Frameworks ist es, sämtliche Funktionalität, die für eine Kommunikation benötigt wird, zur Verfügung zu stellen. Die Kommunikation erfolgt dabei als Direktverbindung über das Real-Time Transport Protokoll (RTP), welches auf UDP basiert. RTP stellt ein standardisiertes Protokoll für die Audio- und Videoübertragung dar, welches 1996 von der Audio-Video Transport Working Group of the Internet Engineering Task Force (IETF) entwickelt wurde. Das Framework ist modular aufgebaut und bietet eine Schnittstelle zur Audiohardware an, ermöglicht den Umgang mit RTP-Paketen, übernimmt das Encodieren und Dekodieren von Audiodaten ins RTP-Protokoll und stellt ein Jitterbuffer zur Verfügung. Sämtliche Funktionen wurden in einer Beispielanwendung integriert und getestet.
\end{abstract}

\tableofcontents

%Teil Inhalt auf, für bessere Modularität und paralleles Schreiben
\chapter{Einleitung}
\section{Was ist OHMComm?}
OHMComm ist ein IT-Projekt der Informatik Fakultät an der technischen Hochschule in Nürnberg. Das Projekt wird im Rahmen des Bachelorstudienganges Informatik umgesetzt und erstreckt sich über zwei Semester. Herr Prof. Dr. M. Teßmann ist Projektinitiator und Projektleiter. Das Ziel des Projektes ist die Erstellung eines plattformunabhängiges Audiokommunikationsframework. Das Framework soll sämtliche Funktionalität, die für eine erfolgreiche Kommunikation zwischen zwei Teilnehmern benötigt wird, zur Verfügung stellen. Das Framework als auch das Projekt tragen den Namen OHMComm. Der Anwendungsfall für die Hochschule ist der Einsatz der Software in Lehrveranstaltungen, jedoch sind auch andere Anwendungsszenarien denkbar, wie z.B. die Integration in externe Softwareanwendungen.

\section{Projektbeschreibung}
Die Projektbeschreibung von Herr Prof. Dr. M. Teßmann beschreibt die Ziele, die im Rahmen des IT-Projektes erreicht werden sollten. Das Framework soll plattformunabhängig und modular aufgebaut sein. Außerdem soll es über eine gute Dokumentation verfügen.
Der obligatorische Funktionsumfang des Audiokommunikationsframework lässt sich in fünf größere Bereiche zerlegen. 
Zum einem wird für die Audioaufnahme und Wiedergabe eine Schnittstelle zur Audiohardware benötigt. 
Der zweite Block beinhaltet die Kodierung und Dekodierung von Audiodaten. Diese Funktionalität darf man keineswegs als optional betrachten, da in der Praxis eine unkomprimierte Audioübertragung meist nicht realisierbar ist. Die meisten Verbindungsleitungen verfügen hierfür nicht über die notwendige Bandbreite. 
Das Versenden und Empfangen von Audiodaten mit dem Real-Time-Transport-Protokol (RTP) auf Basis des User Datagram Protocol (UDP) ist ein weiterer großer Implementierungsblock. RTP stellt ein standardisiertes Protokoll für die Audio- und Videoübertragung dar, welches 1996 von der Audio-Video Transport Working Group of the Internet Engineering Task Force (IETF) entwickelt worden ist \cite{RFC3550}. 
Der vorletzte Block umfasst das Erstellen eines Jitterbuffers. Durch das Versenden von UDP-Paketen kann die Paketreihenfolge beim Empfänger nicht gewährleistet werden. Die Aufgabe des Buffers ist es, durch eine minimale Verzögerung vor dem Abspielen der Audiodaten, die Paketreihenfolge wiederherzustellen, wodurch sich die Audioqualität wesentlich erhöhen lässt.
Der letzte Block beschäftigt sich mit den Themen Test und Bewertung. Sämtliche Funktionen des Frameworks sollen in einer Beispielanwendung validiert werden. Für eine objektive Bewertung des Framesworks sollen statistische Werte gesammelt werden können. Hierbei ist der Aspekt der Performanz zu berücksichtigen, da Latenzen eine wesentliche Rolle in der Audiokommunikation spielen.
	
\section{Aufbau des Berichts}
Der Aufbau des Berichts entspricht den klassischen Phasen der Softwareentwicklung und ist unterteilt in den Kapiteln Anforderungsanalyse, Entwurf, Implementierung, Tests und den Schlussbemerkungen. Das Kapitel Prototypische Voice-over-IP Konsolenanwendung entspricht dem Testkapitel. 
In der Anforderungsanalyse werden aus den Projektzielen die konkreten Anforderungen ermittelt und diese in funktionale und nicht-funktionale Anforderungen klassifiziert. Außerdem wird hier definiert, wie die Ziele und Anforderungen im Rahmen des Projektes interpretiert werden.
In der Entwurfsphase werden für die Umsetzung der Anforderungen Lösungswege erarbeitet, verglichen und bewertet. Außerdem werden in diesem Abschnitt die einzelnen Komponenten des Frameworks spezifiziert sowie die Softwarearchitektur erstellt.
Im Kapitel Implementierung wird der Plan aus der Entwurfsphase umgesetzt. Auf besonders wichtige Implementierungsdetails wird hingewiesen und diese erläutert. 
Die prototypische Voice-over-IP Konsolenanwendung wurde parallel neben den Framework entwickelt und dient ausschließlich zum Testen der Funktionalität. Da diese Anwendung auch ein Teil des Projektes ist, wird diese gesondert, in einem eigenen Kapitel, erörtert. Es wird erklärt wie die Funktionalitäten innerhalb der Anwendung getestet und umgesetzt worden sind.
Im letzten Kapitel wird ein Fazit gezogen. Es wird überprüft, ob alle Ziele und Anforderungen umgesetzt wurden. Außerdem gilt es zu klären, in welchen Umfang die Ziele erreicht wurden und wo es noch Optimierungsbedarf gibt. Im abschließenden Ausblick werden sinnvolle Erweiterungs- und Verbesserungsmöglichkeiten für das Framework vorgeschlagen.
	

\chapter{Anforderungsanalyse}
\section{Funktionale und Nicht-Funktionale Anforderungen}
\section{Datenhaushalt}
\section{Alternative Softwarelösungen}

\chapter{Entwurf}
\section{Projektverwaltung und Werkzeuge}
\section{Build Prozess}
\subsection{CMake}
\subsection{Build unter Linux}
\subsection{Build untre Windows}
\section{Softwarearchitektur}
\subsection{Konfiguration und Verwendung}
\subsection{Audio-Schnittstelle}
\subsection{Verarbeitungskette}
\subsection{Austauschbarkeit und Instanziierung}
\subsection{RTP-Protokoll}
\subsubsection{RTCP-Protokoll}
\subsection{Jitter-Buffer}
\subsection{Netzwerkverbindung}
\section{Konkrete Softwarekomponenten}
\subsection{RTAudio}
\subsection{Opus}
\section{Statistiken}

\chapter{Implementierung}
TODO: Kapitel zur Umsetzung der Konfiguration??

TODO: Umsetzung der passiven Konfiguration. Was/wie wird übertragen? Flowchart/Workflow der Konfiguration (Stalling bis Konfiguration empfangen/gesendet)
\chapter{Prototypische Voice-over-IP Konsolenanwendungen}
\subsection{Ziel der Anwendung}
\subsection{Steuerung}
\subsection{Anwendungen ordnungsgemäß beenden}

\chapter{Schlussbemerkungen}

\section{Ausblick}

\subsection{Anwendungsmöglichkeiten}
OHMComm kann als vollständiges und plattformunabhängiges Audiokommunikationsframework in anderen externen Softwarelösungen integriert werden. Als Schnittstelle nach außen bietet das Framework dafür ein \texttt{OHMComm}-Objekt an, welches erstellt, konfiguriert (bevorzugt über die \texttt{LibraryConfiguration}) und gestartet werden muss. Die gesamte Aktivität des Frameworks wird in separaten Threads (für Audioverarbeitung, Empfangen,, RTCP, ...) ausgelagert. So wird sichergestellt, dass die aufrufende Anwendung nicht blockiert wird. Durch vollständigen Funktionsumfang und wohldefinierten Schnittstellen lässt sich OHMComm einfach in andere Anwendungen integrieren.
Kapitel \ref{prototypProgram} zeigt ein vollständiges Anwendungsbeispiel für die Audioübertragung zwischen zwei Geräten mit OHMComm. Am Beispiel lassen sich sehr einfach die verschiedenen Konfigurationsmodi und Einstellungsmöglichkeiten ausprobieren. Der Code dazu (bestehend aus der Datei \texttt{OHMCommStandalone.cpp}) zeigt ein kurzes und vollständiges Beispiel, wie das Framework richtig eingebunden werden kann.
TODO: erweitern? richtig?

\subsection{Erweiterungsmöglichkeiten}
Aufgrund des modularen Aufbaus des kompletten Frameworks, lässt es sich sehr einfach erweitern. Vorgesehene Erweiterungen für das Framework gliedern sich in die folgenden Kategorien:
\begin{description}
\item[Parameter:] Alle existierenden Parameter können -- soweit einmal registriert -- von allen Audioprozessoren verwendet werden. Ebenso ist es sehr einfach, neue Parameter zu registrieren, die bei der Konfiguration des Frameworks automatisch beachtet und mit Werten versehen werden.
\item[Audioschnittstellen:] Derzeit wird als einzige Schnittstelle zur Audiohardware die Bibliothek RTAudio verwendet. Jedoch kann die Audiobibliothek einfach ausgetauscht werden, indem eine neue Kindklasse von \texttt{AudioHandler} erstellt und zur \texttt{AudioHandlerFactory} hinzugefügt wird, die auf einer anderen Audiobibliothek aufbaut. So könnte z.B. PortAudio, eine plattformunabhängige C-Audioschnittstelle, angebunden werden.
\item[Audioprozessoren:] In die Verarbeitungskette (siehe Abschnitt \ref{processingChain}) können beliebige neue Prozessoren für Audiodaten hinzugefügt werden. Dafür muss nur die neu erstellte Kindklasse von \texttt{AudioProcessor} in \texttt{AudioProcessorFactory} registriert werden und beim Start der Anwendung ausgewählt werden. Ebenso können -- wie bereits erwähnt -- neue Parameter für die Konfiguration der neuen Audioprozessoren registriert werden.
\item[Netzwerkschnittstellen:] Auch neue Schnittstellen für Netzwerkprotokolle, wie TCP, lassen sich durch Erstellen neuer Kindklassen von \texttt{NetworkWrapper} und Ersetzen der Aufrufe von \texttt{UDPWrapper} hinzufügen.
\end{description}
Derzeit sind Erweiterungen zu dem bestehenden Framework in Planung oder bereits in Entwicklung, wie die Konfiguration über das Session Initiation Protocol (SIP), sowie den Audiocodecs G.711 A-law und $\mu$-law, den Standardformaten der digitalen Telefonie.
\\
Im Allgemeinen können in die bestehende Architektur eine Vielzahl an zusätzlichen Funktionalitäten fest oder auch optional hinzugefügt werden. So z.B. weitere Audiocodecs, Hochpass- oder Tiefpassfilter zum Herausfiltern von Rauschen oder Hintergrundgeräuschen, Verstärker für die Lautstärke des abgespielten Signals, Prozessoren, die die Audiokonversation mitschneiden und viele mehr.

\section{Fazit}
Das Hauptziel des Projektes war die Erstellung eines plattformunabhängigen Audiokommunikationsframework. Dieses Ziel wurde im Rahmen des IT-Projektes erreicht, da alle funktionalen und nicht-funktionalen Anforderungen umgesetzt worden sind.
Es wurde eine abstrakte und austauschbare Schnittstelle zur Hardware geschaffen, welche ebenfalls die Architektur zur Audioverarbeitung beinhaltete (Verarbeitungskette). Dadurch konnten andere Klassen aktiv an der Audioverarbeitung teilnehmen. Die konkrete Implementierung dieser abstrakten Klasse basierte dabei auf RtAudio.
Mit Opus wurde ein effizienter Audiocodec eingefügt, welcher die zuvor erstellte Schnittstelle zur Verarbeitungskette nutzte. Das RTP-Protokoll wurde vollständig gemäß dem Standard implementiert und auf Basis des UDP-Protokolls versendet. Empfangene Pakete wurden im Jitter-Buffer sortiert eingefügt. Dieser puffert Pakete bis eine Mindestanzahl erreicht wurde, bevor diese zum Abspielen verfügbar waren. Alle erstellten Funktionalitäten wurden in einer Beispielanwendungen getestet. Die statistische gesammelten Werte bestätigten die Effizienz des Frameworks sowie die Qualität des Opus-Codec. Obwohl das Projekt erfolgreich war, gibt es Kritikpunkte. Das Umsetzen von objektorientierten Prinzipien hätte besser erfolgen können, da die Integration von den angewandten Pattern nicht perfekt war. Der Grund war, dass die entsprechende Vorlesung dazu parallel zum Entwicklungszeitpunkt verlief. Ein weiterer Kritikpunkt war, dass von Anfang an hätten Code-Konventionen angewendet werden sollten für einen einheitlichen sauberen Code. Der Code entspricht einer Mischung von C++11 und C++03.





\bibliographystyle{apalike} % Literaturverzeichnis
\begin{btSect}{./sources/literatur} % mit bibtopic Quellen trennen
\section*{Literaturverzeichnis}
\btPrintCited
\end{btSect}
\begin{btSect}{./sources/online}
\section*{Online-Quellen}
\btPrintCited
\end{btSect}

\end{document}

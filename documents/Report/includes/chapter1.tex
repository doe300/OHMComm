\chapter{Einleitung}
\section{Was ist OHMComm?}
OHMComm ist ein IT-Projekt der Informatik Fakultät an der technischen Hochschule in Nürnberg. Das Projekt wird im Rahmen des Bachelorstudienganges Informatik umgesetzt und erstreckt sich über zwei Semester. Herr Prof. Dr. M. Teßmann ist Projektinitiator und Projektleiter. Das Ziel des Projektes ist die Erstellung eines plattformunabhängiges Audiokommunikationsframework. Das Framework soll sämtliche Funktionalität, die für eine erfolgreiche Kommunikation zwischen zwei Teilnehmern benötigt wird, zur Verfügung stellen. Das Framework als auch das Projekt tragen den Namen OHMComm. Der Anwendungsfall für die Hochschule ist der Einsatz der Software in Lehrveranstaltungen, jedoch sind auch andere Anwendungsszenarien denkbar, wie z.B. die Integration in externe Softwareanwendungen.

\section{Projektbeschreibung}
Die Projektbeschreibung von Herr Prof. Dr. M. Teßmann beschreibt die Ziele, die im Rahmen des IT-Projektes erreicht werden sollten. Das Framework soll plattformunabhängig und modular aufgebaut sein. Außerdem soll es über eine gute Dokumentation verfügen.
Der obligatorische Funktionsumfang des Audiokommunikationsframework lässt sich in fünf größere Bereiche zerlegen. 
Zum einem wird für die Audioaufnahme und Wiedergabe eine Schnittstelle zur Audiohardware benötigt. 
Der zweite Block beinhaltet die Kodierung und Dekodierung von Audiodaten. Diese Funktionalität darf man keineswegs als optional betrachten, da in der Praxis eine unkomprimierte Audioübertragung meist nicht realisierbar ist. Die meisten Verbindungsleitungen verfügen hierfür nicht über die notwendige Bandbreite. 
Das Versenden und Empfangen von Audiodaten mit dem Real-Time-Transport-Protokol (RTP) auf Basis des User Datagram Protocol (UDP) ist ein weiterer großer Implementierungsblock. RTP stellt ein standardisiertes Protokoll für die Audio- und Videoübertragung dar, welches 1996 von der Audio-Video Transport Working Group of the Internet Engineering Task Force (IETF) entwickelt worden ist \cite{RFC3550}. 
Der vorletzte Block umfasst das Erstellen eines Jitterbuffers. Durch das Versenden von UDP-Paketen kann die Paketreihenfolge beim Empfänger nicht gewährleistet werden. Die Aufgabe des Buffers ist es, durch eine minimale Verzögerung vor dem Abspielen der Audiodaten, die Paketreihenfolge wiederherzustellen, wodurch sich die Audioqualität wesentlich erhöhen lässt.
Der letzte Block beschäftigt sich mit den Themen Test und Bewertung. Sämtliche Funktionen des Frameworks sollen in einer Beispielanwendung validiert werden. Für eine objektive Bewertung des Framesworks sollen statistische Werte gesammelt werden können. Hierbei ist der Aspekt der Performanz zu berücksichtigen, da Latenzen eine wesentliche Rolle in der Audiokommunikation spielen.
	
\section{Aufbau des Berichts}
Der Aufbau des Berichts entspricht den klassischen Phasen der Softwareentwicklung und ist unterteilt in den Kapiteln Anforderungsanalyse, Entwurf, Implementierung, Tests und den Schlussbemerkungen. Das Kapitel Prototypische Voice-over-IP Konsolenanwendung entspricht dem Testkapitel. 
In der Anforderungsanalyse werden aus den Projektzielen die konkreten Anforderungen ermittelt und diese in funktionale und nicht-funktionale Anforderungen klassifiziert. Außerdem wird hier definiert, wie die Ziele und Anforderungen im Rahmen des Projektes interpretiert werden.
In der Entwurfsphase werden für die Umsetzung der Anforderungen Lösungswege erarbeitet, verglichen und bewertet. Außerdem werden in diesem Abschnitt die einzelnen Komponenten des Frameworks spezifiziert sowie die Softwarearchitektur erstellt.
Im Kapitel Implementierung wird der Plan aus der Entwurfsphase umgesetzt. Auf besonders wichtige Implementierungsdetails wird hingewiesen und diese erläutert. 
Die prototypische Voice-over-IP Konsolenanwendung wurde parallel neben den Framework entwickelt und dient ausschließlich zum Testen der Funktionalität. Da diese Anwendung auch ein Teil des Projektes ist, wird diese gesondert, in einem eigenen Kapitel, erörtert. Es wird erklärt wie die Funktionalitäten innerhalb der Anwendung getestet und umgesetzt worden sind.
Im letzten Kapitel wird ein Fazit gezogen. Es wird überprüft, ob alle Ziele und Anforderungen umgesetzt wurden. Außerdem gilt es zu klären, in welchen Umfang die Ziele erreicht wurden und wo es noch Optimierungsbedarf gibt. Im abschließenden Ausblick werden sinnvolle Erweiterungs- und Verbesserungsmöglichkeiten für das Framework vorgeschlagen.
	
